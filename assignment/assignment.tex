\documentclass[11pt,a4paper]{scrartcl}

% - Base packages -
\usepackage[utf8]{inputenc}
\usepackage[T1]{fontenc}
\usepackage[english]{babel}
\usepackage[a4paper, top=2.5cm, bottom=2.5cm, left=2.8cm, right=2.8cm, headheight=36.5pt, footskip=1.5cm]{geometry}
\usepackage{amsmath,amssymb}
\usepackage{graphicx}
\usepackage{enumitem}
\usepackage{hyperref}
\usepackage[automark,headsepline]{scrlayer-scrpage}
\usepackage{xcolor}
\usepackage{tcolorbox}
\usepackage{listings}
\usepackage{microtype}
\usepackage{tikz}  % For enhanced graphics and borders
\usetikzlibrary{shadows,patterns}

% Professional monospace font setup
\usepackage{inconsolata}  % Clean, crisp monospace font
\usepackage[scaled=1.0]{helvet}  % For sans-serif headers
\renewcommand{\familydefault}{\ttdefault}  % Use monospace as main font
\usepackage[libertine]{newtxmath}  % Math font that pairs well with monospace

% Professional grayscale color scheme
\definecolor{titlecolor}{RGB}{20, 20, 20}
\definecolor{sectioncolor}{RGB}{40, 40, 40}
\definecolor{accentcolor}{RGB}{60, 60, 60}
\definecolor{rulecolor}{RGB}{100, 100, 100}
\definecolor{lightgray}{RGB}{245, 245, 245}
\definecolor{codebg}{RGB}{250, 250, 250}
\definecolor{codehighlight}{RGB}{240, 240, 245}
\definecolor{shadowcolor}{RGB}{220, 220, 220}

% Custom command for inline code paths with subtle highlight
\newcommand{\codepath}[1]{\colorbox{codehighlight}{\texttt{#1}}}

% Section divider command
\newcommand{\sectiondivider}{%
  \vspace{0.2em}%
  \noindent\textcolor{rulecolor}{\rule{0.3\textwidth}{0.4pt}}%
  \vspace{0.3em}%
}

% Hyperref setup - understated
\hypersetup{
    colorlinks=true,
    linkcolor=black,
    urlcolor=accentcolor,
    citecolor=black,
    pdfborder={0 0 0}
}

% Code listing setup - minimal and crisp
\lstset{
    language=Python,
    basicstyle=\ttfamily\small,
    breaklines=true,
    frame=none,
    backgroundcolor=\color{codebg},
    xleftmargin=1em,
    framexleftmargin=0em,
    columns=fullflexible,
    showstringspaces=false,
    keywordstyle=\bfseries,
    commentstyle=\color{accentcolor}\itshape,
    stringstyle=\itshape,
    numbers=left,
    numberstyle=\tiny\color{rulecolor},
    numbersep=1em
}

% Professional tcolorbox styles with shadows
\tcbuselibrary{listings,breakable,skins}

\tcbset{
    modern/.style={
        enhanced,
        sharp corners,
        colback=codebg,
        colframe=rulecolor,
        boxrule=0.5pt,
        arc=0mm,
        top=8pt,
        bottom=8pt,
        left=12pt,
        right=12pt,
        breakable,
        before skip=10pt,
        after skip=10pt,
        drop fuzzy shadow=shadowcolor
    }
}

% Section formatting - minimal and professional with accent bars
\setkomafont{section}{\ttfamily\LARGE\bfseries\color{sectioncolor}}
\setkomafont{subsection}{\ttfamily\Large\bfseries\color{sectioncolor}}
\setkomafont{subsubsection}{\ttfamily\large\bfseries\color{accentcolor}}

% Add subtle bars before sections
\RedeclareSectionCommand[
  beforeskip=1.0\baselineskip,
  afterskip=.5\baselineskip
]{section}

\RedeclareSectionCommand[
  beforeskip=0.8\baselineskip,
  afterskip=.3\baselineskip
]{subsection}

% List spacing with improved bullets
\setlist{nosep, leftmargin=1.8em}
\setlist[itemize,1]{label={\textcolor{rulecolor}{\small\textbullet}}}
\setlist[enumerate,1]{label={\textcolor{sectioncolor}{\arabic*.}}}

% Paragraph spacing - compact layout
\setlength{\parskip}{0.4em}
\setlength{\parindent}{0pt}

\setlength{\footheight}{25pt}

\clearpairofpagestyles

% Enhanced header with subtle line
\chead{%
  \raisebox{-2pt}{\includegraphics[height=1.1cm]{logo.png}}
}
\cfoot{%
  \vspace{-0.5em}%
  \textcolor{rulecolor}{\rule{4cm}{0.3pt}}\\[0.3em]%
  \ttfamily\small\pagemark%
}
\renewcommand*{\titlepagestyle}{scrheadings}

\addtokomafont{title}{\vspace*{-2em}}
\addtokomafont{author}{\vspace*{-0.5em}}
\addtokomafont{date}{\vspace*{-0.5em}}

% - Title and authors -
\title{\ttfamily\Huge \textbf{Particle Swarm Optimization}\\[0.3em]
\LARGE Learning Swarm Intelligence with RL}

\author{
\ttfamily Collective Intelligence - Spring 2026
}
\date{\ttfamily Teams of 2-3 students | 6-8 weeks}

\begin{document}
\maketitle

\vspace{-3em}

\begin{center}
    \includegraphics[width=0.6\textwidth]{pso.png}
\end{center}

% =========================
% OVERVIEW
% =========================
\section*{Overview}

This project explores \textbf{learned particle swarm optimization (PSO)} using multi-agent reinforcement learning. Instead of hand-tuning hyperparameters like inertia weight $\omega$, cognitive coefficient $c_1$, and social coefficient $c_2$, agents learn to dynamically adjust these parameters based on the current optimization landscape.

Your task is to extend the base implementation with \textbf{one of three research directions} below. Each task involves implementing new functionality, running rigorous experiments, and documenting results with visualizations and analysis.

\begin{tcolorbox}[modern, title=\ttfamily\textbf{What is already provided}]
\begin{itemize}[leftmargin=1.2em, itemsep=4pt]
    \item \codepath{src/envs/env.py} --- TorchRL-based multi-agent PSO environment
    \item \codepath{src/envs/dynamic\_functions.py} --- Static and dynamic benchmark functions (Sphere, Rastrigin, Eggholder, etc.)
    \item \codepath{src/main.py} --- Hydra-based training loop with PPO
    \item \codepath{src/eval.py} --- Evaluation with 2D/3D visualization and GIF generation
    \item \codepath{src/visualizer.py} --- SwarmVisualizer class for particle animation
    \item Docker containerization for reproducibility
\end{itemize}
\end{tcolorbox}

\begin{tcolorbox}[modern, title=\ttfamily\textbf{Teams \& Timeline}]
\begin{itemize}[leftmargin=1.2em, itemsep=4pt]
    \item \textbf{3 teams} of 2-3 students each
    \item \textbf{Deadline}: 6-8 weeks from start
    \item \textbf{Each team chooses ONE task} from the three options below
\end{itemize}
\end{tcolorbox}

\sectiondivider

% =========================
% TASK OPTIONS
% =========================
\section*{Task Options}

Each team selects \textbf{one task}. All tasks have comparable scope and grading weight.

\subsection*{Task 1: Communication Topologies \& Information Flow}

\textbf{Research Question}: \textit{How do different swarm communication topologies (global best, local best, dynamic neighborhoods) affect convergence and diversity in learned PSO?}

\subsubsection*{Hypothesis}
Restricting information flow through local topologies (ring, von Neumann, dynamic k-nearest) will maintain diversity longer, improving performance on multimodal landscapes, while global best topology converges faster on unimodal functions.

\subsubsection*{Implementation Requirements}
\begin{enumerate}[leftmargin=1.4em]
    \item \textbf{Topology System} (\codepath{src/envs/topology.py} --- new file):
    \begin{itemize}
        \item Implement: Global Best (gBest), Local Best (lBest) with ring topology
        \item Implement: Von Neumann (grid) neighborhood, k-Nearest dynamic neighbors
        \item Topology abstraction: \texttt{Topology.get\_neighbors(particle\_id)} interface
        \item Allow topology changes during optimization (learned or scheduled)
    \end{itemize}
    
    \item \textbf{Environment Extension} (\codepath{src/envs/env.py} modification):
    \begin{itemize}
        \item Add topology-aware observation: include neighbor personal bests
        \item Modify reward to use neighborhood best instead of global best
        \item Config-driven topology selection via Hydra
    \end{itemize}
    
    \item \textbf{Diversity Metrics} (\codepath{src/eval/diversity.py} --- new file):
    \begin{itemize}
        \item Position diversity: mean pairwise distance, convex hull volume
        \item Velocity diversity: velocity alignment coefficient
        \item Information spread: how fast does best solution propagate?
    \end{itemize}
\end{enumerate}

\subsubsection*{Technical Guidance}
\begin{itemize}
    \item Use adjacency matrices for efficient neighbor queries
    \item Dynamic k-nearest: recompute neighbors every N steps (configurable)
    \item Configs: \codepath{src/configs/topology/} with \texttt{global.yaml}, \texttt{ring.yaml}, \texttt{knearest.yaml}
    \item Consider graph neural network observations for topology-aware policies
\end{itemize}

\subsubsection*{Expected Deliverables}
\begin{itemize}
    \item Working topology system with 4+ topology types
    \item Diversity metrics integrated into evaluation pipeline
    \item Comparison: 4 topologies $\times$ 3 landscapes (unimodal, multimodal, dynamic) $\times$ 5 seeds
    \item 4-6 plots: convergence curves, diversity over time, final fitness distributions, topology comparison heatmaps
    \item 2-3 GIFs: side-by-side topology comparison on same function
    \item Analysis: when does local topology help? Trade-offs between exploration and exploitation?
\end{itemize}

\sectiondivider

\subsection*{Task 2: Curriculum Learning \& Adaptive Difficulty}

\textbf{Research Question}: \textit{Can progressively increasing optimization difficulty (dimensionality, function complexity, landscape dynamics) improve sample efficiency and generalization?}

\subsubsection*{Hypothesis}
Starting training on simpler problems (low dimensions, unimodal functions) and gradually introducing complexity will produce policies that generalize better to hard problems compared to training directly on difficult tasks.

\subsubsection*{Implementation Requirements}
\begin{enumerate}[leftmargin=1.4em]
    \item \textbf{Curriculum Manager} (\codepath{src/training/curriculum.py} --- new file):
    \begin{itemize}
        \item Dimension curriculum: train 2D $\rightarrow$ 5D $\rightarrow$ 10D $\rightarrow$ 30D
        \item Function curriculum: Sphere $\rightarrow$ Rosenbrock $\rightarrow$ Rastrigin $\rightarrow$ Eggholder
        \item Dynamics curriculum: static $\rightarrow$ slow dynamics $\rightarrow$ fast dynamics
        \item Progression criteria: performance threshold, training steps, or learned
    \end{itemize}
    
    \item \textbf{Generalization Testing} (\codepath{src/eval/generalization.py} --- new file):
    \begin{itemize}
        \item Zero-shot transfer: evaluate on unseen dimensions/functions
        \item Performance gap: curriculum-trained vs direct-trained on hard problems
        \item Learning curves: sample efficiency comparison
    \end{itemize}
    
    \item \textbf{Adaptive Difficulty} (bonus extension):
    \begin{itemize}
        \item Automatic difficulty adjustment based on agent performance
        \item Multi-armed bandit for curriculum stage selection
        \item Self-paced learning: agents request harder problems
    \end{itemize}
\end{enumerate}

\subsubsection*{Technical Guidance}
\begin{itemize}
    \item Dimension-agnostic policy: use attention or recurrent networks for variable input sizes
    \item Function encoding: one-hot, learned embeddings, or meta-learning
    \item Configs: \codepath{src/configs/curriculum/} with \texttt{dimension.yaml}, \texttt{function.yaml}, \texttt{combined.yaml}
    \item Consider domain randomization as baseline comparison
\end{itemize}

\subsubsection*{Expected Deliverables}
\begin{itemize}
    \item Working curriculum system with 3+ curriculum types
    \item Generalization evaluation across unseen problems
    \item Comparison: curriculum vs direct training on final tasks (5 seeds each)
    \item 4-6 plots: learning curves, generalization matrices, curriculum progression, sample efficiency
    \item 2-3 GIFs: policy behavior on easy vs hard problems
    \item Analysis: optimal curriculum ordering? When does curriculum help/hurt? Transfer limits?
\end{itemize}

\sectiondivider

\subsection*{Task 3: Diversity-Preserving Reward Shaping}

\textbf{Research Question}: \textit{Can auxiliary rewards for swarm diversity prevent premature convergence and improve global optimization on deceptive landscapes?}

\subsubsection*{Hypothesis}
Adding intrinsic motivation rewards for maintaining swarm diversity (position spread, velocity diversity, exploration bonuses) will improve performance on multimodal and deceptive functions where standard PSO suffers from premature convergence.

\subsubsection*{Implementation Requirements}
\begin{enumerate}[leftmargin=1.4em]
    \item \textbf{Diversity Rewards} (\codepath{src/envs/diversity\_reward.py} --- new file):
    \begin{itemize}
        \item Position entropy: reward for spread in search space
        \item Novelty bonus: reward for visiting unexplored regions (archive-based)
        \item Velocity alignment penalty: discourage all particles moving same direction
        \item Scalarization: $r = \alpha \cdot r_{\text{fitness}} + (1-\alpha) \cdot r_{\text{diversity}}$
    \end{itemize}
    
    \item \textbf{Adaptive Weighting}:
    \begin{itemize}
        \item Schedule: high diversity weight early, decay over optimization
        \item Performance-based: increase diversity when stuck, decrease when improving
        \item Learned: let the policy learn to balance via multi-objective heads
    \end{itemize}
    
    \item \textbf{Premature Convergence Detection} (\codepath{src/eval/convergence\_analysis.py}):
    \begin{itemize}
        \item Detect: when does swarm collapse? Distance to centroid metrics
        \item Compare: baseline vs diversity-rewarded policies
        \item Measure: escape rate from local optima
    \end{itemize}
\end{enumerate}

\subsubsection*{Technical Guidance}
\begin{itemize}
    \item Position entropy: $H = -\sum_i p_i \log p_i$ over discretized search space bins
    \item Novelty archive: store visited positions, reward distance to k-nearest archive points
    \item Configs: \codepath{src/configs/reward/} with \texttt{baseline.yaml}, \texttt{diversity.yaml}, \texttt{adaptive.yaml}
    \item Test on deceptive functions: Rastrigin, Schwefel, multimodal custom landscapes
\end{itemize}

\subsubsection*{Expected Deliverables}
\begin{itemize}
    \item 3+ diversity reward mechanisms with configurable weights
    \item Adaptive weighting system (at least 2 strategies)
    \item Comparison: baseline vs diversity rewards on 4+ functions (5 seeds each)
    \item 4-6 plots: diversity over time, convergence detection, $\alpha$ sensitivity, local optima escape rates
    \item 2-3 GIFs: baseline (premature convergence) vs diversity-rewarded (sustained exploration)
    \item Analysis: optimal diversity-fitness trade-off? Function-dependent tuning? Failure modes?
\end{itemize}

\sectiondivider

\section*{Shared Requirements (All Tasks)}

All teams must follow these common guidelines to ensure quality and reproducibility.

\begin{tcolorbox}[modern]
\textbf{Code Quality}
\begin{itemize}
    \item Follow existing code structure (TorchRL patterns, Hydra configs)
    \item Update relevant folder READMEs documenting new modules
    \item Add 2-3 unit tests for core functionality
    \item Hydra configs for all experiments
\end{itemize}

\textbf{Experiments}
\begin{itemize}
    \item Run on both \textbf{low-dimensional} (2D-5D) and \textbf{high-dimensional} (10D-30D) problems
    \item Test on multiple function types: unimodal (Sphere), multimodal (Rastrigin), dynamic
    \item Use fixed seeds for reproducibility (report all seeds in README)
    \item Minimum 5 seeds per experimental condition
\end{itemize}

\textbf{Documentation}
\begin{itemize}
    \item Update root \texttt{README.md} with new ``Semester Contribution'' section:
    \begin{itemize}
        \item Research question \& hypothesis
        \item Implementation summary (what was added/modified)
        \item Key results (embed 2-3 key plots/GIFs)
        \item Conclusions \& limitations
        \item Future work
    \end{itemize}
    \item Keep it concise (500-800 words max)
\end{itemize}
\end{tcolorbox}

% =========================
% GRADING RUBRIC
% =========================
\section*{Grading Rubric}

\begin{tcolorbox}[modern, title=\ttfamily\textbf{Total: 100 points}]
Each task has the same grading structure. All requirements must use Hydra configs and be reproducible.
\end{tcolorbox}

\vspace{0.5em}

\subsection*{Implementation (40 pts)}
\begin{itemize}
    \item Core functionality working (new modules/classes follow codebase conventions) \dotfill \textbf{20 pts}
    \item Integration with existing system (configs, training loops, evaluation) \dotfill \textbf{10 pts}
    \item Code quality (documentation, tests, readable) \dotfill \textbf{10 pts}
\end{itemize}

\subsection*{Experiments (30 pts)}
\begin{itemize}
    \item Rigorous experimental design (controls, baselines, multiple seeds) \dotfill \textbf{10 pts}
    \item Sufficient scale (low-D + high-D, 5+ conditions) \dotfill \textbf{10 pts}
    \item Reproducibility (fixed seeds, configs, clear instructions) \dotfill \textbf{10 pts}
\end{itemize}

\subsection*{Results \& Analysis (20 pts)}
\begin{itemize}
    \item Clear visualizations (4-6 plots + 2-3 GIFs with captions) \dotfill \textbf{10 pts}
    \item Insightful analysis (hypothesis testing, failure cases, limitations) \dotfill \textbf{10 pts}
\end{itemize}

\subsection*{Documentation (10 pts)}
\begin{itemize}
    \item Semester Contribution in root README (clear, well-structured) \dotfill \textbf{6 pts}
    \item Updated folder READMEs for modified modules \dotfill \textbf{4 pts}
\end{itemize}

\subsection*{Bonus (up to +10 pts)}
\begin{itemize}
    \item Novel extension beyond task requirements (e.g., new functions, theoretical analysis, hybrid approaches) \dotfill \textbf{+5 pts}
    \item Exceptional results (paper-quality plots, surprising insights, strong baselines) \dotfill \textbf{+5 pts}
\end{itemize}

\sectiondivider

\section*{Presentation}

Brief 10-minute presentation with slides covering: (1) Research question, (2) Implementation highlights, (3) Key results (3-4 plots/GIFs), (4) Failure analysis, (5) Future work. Live demo or pre-recorded GIF required.

\section*{Submission}

\begin{tcolorbox}[modern]
\begin{itemize}
    \item \textbf{GitHub}: Push all code, configs, and updated READMEs to repository
    \item \textbf{Canvas}: Submit PDF slides and link to GitHub branch/PR
    \item \textbf{Artifacts}: Include plots/GIFs in \codepath{images/semester\_contribution/}
    \item \textbf{Team}: Document individual contributions in README
\end{itemize}
\end{tcolorbox}

\vspace{2em}

\noindent\rule{\textwidth}{0.5pt}

\vspace{0.5em}
\noindent\textbf{Prepared by:} Tamás Takács \\
\textbf{Semester:} 2025/26/2

\vspace{1em}
\begin{center}
    \footnotesize\color{accentcolor} Licensed under \href{https://creativecommons.org/licenses/by-nc-nd/4.0/}{CC BY-NC-ND 4.0}. \copyright~Tamás Takács, 2025.
\end{center}
 
\end{document}
